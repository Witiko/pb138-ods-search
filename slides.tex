\documentclass{beamer}
% Boilerplate.
\usepackage[czech,main=english]{babel}
\usepackage[utf8]{inputenc}
\usetheme{fibeamer}
\usepackage[tightLists=false]{markdown}
% Metadata.
\title{Searching Open Document Spreadsheets}
\subtitle{Java Implementation, Team B}
\author{M. Majer, S. Markosová, V. Novotný, and P. Kratochvíla}
% Markup.
\let\acro\MakeUppercase
% Document.
\begin{document}
\frame{\maketitle}
\begin{markdown*}{
  hybrid, cacheDir = /tmp/markdown,
  smartEllipses, startNumber = false,
  renderers={
    headingOne = {\frametitle{#1}},
    headingTwo = {\framesubtitle{#1}},
    emphasis = {\alert{#1}},
    image = {\begin{center}
      \includegraphics[width=.8\textwidth]{#2}
    \end{center}}
  }
}
\begin{darkframes}
\begin{frame}

# Analysis
## The Open Document Format specification

\acro{ods} documents come in two flavours

  1. _\acro{xml} documents_ and
  2. _\acro{zip} archives_ containing the file `content.xml`.

and are usually extended.

  * The spec permits _the inclusion of arbitrary elements and attributes_
    beyond what is specified by the Relax \acro{NG} schema.

_Holes_ in tables are represented by a single cell with the
`number-rows` and `columns-repeated` attributes.

  * This makes it difficult to tell the coordinates of a cell without
    pre-processing the entire document beforehand.

\end{frame}
\begin{frame}

# Design
## The high-level overview

There are three packages.

  * _`common`_ -- The effective code for processing ODS documents,
    * Assigned to Matej Majer and Svetlana Markosová.
  * _`cli`_ -- The command-line interface of the program,
    * Assigned to Petr Kratochvíla.
  * _`gui`_ -- The graphical user interface of the program.
    * Assigned to Vít Novotný.

\end{frame}
\begin{frame}

# The _`common`_ package
## The first draft

 1. Deserialize the input \acro{xml} files to a tree structure of Java objects
    that correspond to the pruned \acro{dom} tree with coordinates added to
    the cells.
 2. Perform any further queries on this internal java object representation.

Straightforward, but makes sparse use of \acro{xml} technologies. Therefore,
this draft was scrapped.

\end{frame}
\begin{frame}

# The _`common`_ package
## The second draft

 1. Convert the input \acro{xml} files to an intermediary \acro{xml} format
    with resolved cell coordinates via a \acro{xslt} transformation.
 2. Unlike the extended \acro{ods} documents, this intermediary format can
    now be validated against a \acro{xml} schema.
 3. Perform any further queries on this internal \acro{xml} representation
    via XPath.

Extensive use of \acro{xml} technologies. This draft was implemented.

\end{frame}
\begin{frame}

# The _`gui`_ package

\vspace{-1em}

 ![gui](gui.png "The \acro{swing} \acro{gui}")

\end{frame}
\begin{frame}

# The _`cli`_ package

    usage: java -jar pb138-ods-search.jar
                [-h] [-I] [-x] -s <arg> [FILE]...

       -h,----help             Help
       -I,----ignore-case      Use case-insensitive search
       -s,----string <arg>     String to be used for search
       -x,----exact-match      Use exact match search

\end{frame}
\end{darkframes}
\end{markdown*}
\end{document}
